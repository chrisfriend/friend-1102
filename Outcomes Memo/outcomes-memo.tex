\documentclass[11pt, oneside]{amsart}	%defines this as an article
\usepackage{chrisfriend-comp} %provides formatting declarations for page, headers, figures, textcolor, comments, and bibliographic styles
\usepackage{chrisfriend-OTF-support} %provides support for OTF system fonts; incompatible with latex, rtf2latex, & ht4latex
%\usepackage[utf8]{inputenc} %support for smallamp?

%\usepackage{tabularx}
\usepackage{tabulary} % allows for the tables I make rubrics with
%\usepackage{supertabular}
\usepackage{xtab} % allows tables to span pages
\usepackage{booktabs} % allows fancy lines in tables
\usepackage{rotating} % allows landscape tables
\usepackage{lscape} % allows rotated longtables
\usepackage{multirow} % allows rowspanning
\usepackage{enumitem} % helps with the overview
\usepackage{acronym}
\input{acronyms}
%\usepackage{paralist}
%\usepackage{draftwatermark}

\title[Outcomes Memo]{Assignment Sheet: Outcomes Memo}
\chead{\scriptsize{\MakeUppercase{Outcomes Memo}}}
\lhead{\scriptsize{\textsc{enc 1102}}}

\begin{document}
%\bibliographystyle{abbrv}
\thispagestyle{empty}
%\vspace{-3in}

%\twocolumn[
\begin{center}
\huge
{\includegraphics[scale=.4]{pegasus.pdf}}

\textbf{Assignment Sheet: Outcomes Memo}

{\normalsize Chris Friend • \textsc{enc1102} • Summer B 2013}
\end{center}
\vspace{\baselineskip}
%] %Use for column-spanning the title

\section{Purpose} % (fold)
\label{sec:purpose}
How well you achieve the course outcomes listed in the syllabus determines your level of success for the semester. This assignment asks you to consider your progress toward that goal by having you assess how well you are achieving those outcomes as the semester progresses. By pausing to reflect on the course outcomes at the end of each assignment, you keep those outcomes fresh in your mind, making it more likely you will work toward them and other assignments, and making the Course Audit at the end of the semester much easier to write because you will be used to writing about the outcomes.

In short, this assignment is practice for the cover letter for your final portfolio. Use this memo as an opportunity to document how your work displays evidence of the course outcomes.
% section purpose (end)


\section{Procedure} % (fold)
\label{sec:procedure}
\begin{enumerate}
	\item Review the list of course outcomes on the syllabus and determine how much progress you have made toward those outcomes with the assignment you have just completed. Note: You won't address all the outcomes with any one assignment, and you might only make progress toward an outcome, rather than completing it. By the end of the semester, you should have done everything on the list; for each assignment, see what did and did not get included.
	\item Note specific examples of how you could prove you worked toward those outcomes. This evidence might come from quotes of your assignment submission, references to entire sections, or descriptions of processes you went through while creating the document. Remember, many of the outcomes relate to your \emph{thinking} that leads to better writing. You may need to describe your thought process, rather than quoting something from your work.
	\item Use Google Docs or your favorite word processor to compose a memo\footnote{Use the built-in templates for your word processor, or find a memo template online for use with Google Docs. Don't spend time we creating this format when others have already done it for you.} addressed to your instructor that briefly lists any of the course outcomes that apply to the work you've done on the assignment and provides an explanation of how you made progress toward or achieved each outcome through this assignment.
\end{enumerate}



% section process (end)

\section{Evaluation} % (fold)
\label{sec:evaluation}
Because this memo is merely an attachment to a real assignment, it is not being very critically evaluated. Your instructor will be looking for \emph{clarity} and \emph{supporting evidence} in your memos. Treat these documents more like practice exercises and less like final papers that require delicate proofreading.
% section evaluation (end)

\end{document}

Example:
For example, the first outcome listed is “read, analyze, and respond to difficult texts.” as part of the Brainstorming Audit assignment, I read several articles to determine what sort of topic I wanted to research. I didn't actually analyze any of those texts, but I certainly responded to them by choosing a topic and thinking a little more about the ideas expressed in each article. For that assignment's Outcomes Memo, I could list this first outcome and say that I ``read several difficult texts that helped me find an interesting research topic. These texts were difficult because I understood the general point they were making without being able to comfortably understand the specific details of their arguments because I didn't get where they were coming from. While I didn't yet analyze these difficult texts, I certainly responded to them by determining which ones seemed interesting and choosing a topic as a result.''




\section{Formatting} % (fold)
\label{sec:formatting}
This assignment is obviously not a traditional paper. Therefore, you should avoid the normal paper-formatting approach. Consider using your software's built-in templates for memos. Whatever solution you use, be sure to include the following traditional memo elements:
\begin{itemize}
	\item \textbf{single}-spaced lines
	\item professional typeface
	\item clear section headings
	\item lists, where appropriate
	\item routing information above a divider
\end{itemize}
% section formatting (end)

\end{document}
