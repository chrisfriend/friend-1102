\documentclass[12pt,oneside]{amsart}	%defines this as an article
\usepackage{chrisfriend-comp} %provides formatting declarations for page, headers, figures, textcolor, comments, and bibliographic styles
\usepackage{chrisfriend-OTF-support} %provides support for OTF system fonts; incompatible with latex, rtf2latex, & ht4latex
%\usepackage[utf8]{inputenc} %support for smallamp?

%\usepackage{draftwatermark}

%\usepackage{tabularx}
\usepackage{tabulary} % allows for the tables I make rubrics with
%\usepackage{supertabular}
\usepackage{xtab} % allows tables to span pages
\usepackage{booktabs} % allows fancy lines in tables
\usepackage{rotating} % allows landscape tables
\usepackage{lscape} % allows rotated longtables
\usepackage{multirow} % allows rowspanning
\usepackage{enumitem} % helps with the overview
%\usepackage{paralist}

\title[Brainstorming Audit]{Assignment Sheet: Brainstorming Audit}
\chead{\scriptsize{\MakeUppercase{Brainstorming Audit}}}

\begin{document}
%\bibliographystyle{abbrv}
\thispagestyle{empty}

\vspace{-2in}
%\twocolumn[
\begin{center}
\huge
{\includegraphics[scale=.4]{pegasus.pdf}}

\textbf{Assignment Sheet: Brainstorming Audit}

{\normalsize Chris Friend • \textsc{enc1102} • Spring 2013}
\end{center}
\vspace{1.5\baselineskip}
%] %Use for column-spanning the title

\section{Background} % (fold)
\label{sec:background}
The readings you have done, writing journal entries you have made, and class discussions you have participated in have all been designed to get you thinking about possible questions or problems you might want to research. For this assignment, you will choose one issue to research and document your decision. \textbf{Please note: this assignment is not binding, and you may refine or revise your decision as time goes on.} However, it is designed to force you to think through the feasibility of—and your interest level in—a potential direction for your research.
% section background (end)

\section{Purpose} % (fold)
\label{sec:purpose}
To ensure you have thoroughly considered the subject you would like to research, you will create a two-page document that shows you can do the following:
\begin{enumerate}
	\item Identify a problem to be researched that a) you are concerned about; b)  comes from your own experiences or observations; and c) relates to reading, writing, language, or literacy.
	\item  Determine how and why you are interested in this problem, likely based on how it relates to your past experiences. Include an anecdote about a single noteworthy experience, or identify several related experiences or observations that prompt your interest.
	\item  Specify what you would like to learn about this problem, question, or experience.
	\item Explain \emph{why} you would like to learn more about this issue.
	\item  Document the authors from our in-class readings who relate to your problem or question. Cite and document that relation by using an appropriate citation style (such as \textsc{mla} or \textsc{apa}). This means your document will have a Works Cited or References page.
\end{enumerate}

 It is important to note what this document does \emph{not} do. It is \emph{not} a summary or synthesis of the reading responses or journal entries you have already created. Those documents were your brainstorming process. This assignment is to identify what you have concluded from that process and provide more in-depth thinking about the problem or question you would like to research.
% section purpose (end)

\section{Evaluation} % (fold)
\label{sec:evaluation}
This assignment exists primarily to create a starting point for your research project. It is not a final paper, so it need not have the feel of a research report. Instead, it should illustrate that you have thought carefully about material from this course so far. Be sure that your ideas are appropriately organized and edited so that your thinking comes through clearly. Evaluation will be based on the rubric shown in Table~\ref{tab:rubric}.
% section rubric (end)

\section{Procedure} % (fold)
\label{sec:procedure}
Before you start auditing your brainstorming process, get your thoughts in order. Your reading responses, writing log, and history as a writer should be in mind as you draft this assignment.
\begin{enumerate}
	\item Identify a problem, issue, or concern you have related to writing, reading, language, or literacy. Choose something you would genuinely like to learn more about.
	\item Question yourself to determine why this topic interests you or why you want to know more about it. You might want to discuss it with a friend or do some freewriting to get your thoughts moving.
	\item Come up with a finish line. What could you learn that would satisfy your curiosity? What is your goal in this project?
	\item Now write. Put the ideas from the items above in a two-page document that puts your ideas in writing, helps me understand your thinking/goals, and shows your classmates where you're heading with your project.
\end{enumerate}
% section procedure (end)

\begin{table}[b]
	\caption{Evaluation of Brainstorming Audit}\label{tab:rubric}
\begin{tabulary}{\textwidth}{rLLLL}
	\toprule  & \textbf{\textsc{Depth of Thought}} & \textbf{\textsc{Rationale}} & \textbf{\textsc{Question or Problem}} & \textbf{\textsc{Readings}}\\
\midrule	\textbf{Excellent} & Shows clear, considered thinking about research topic and goals & Combination of personal interest and thinking show natural choice of issue & Clearly identifies topic as a provoking question or pressing problem to be studied & Problem/question shown to build naturally from class readings \\
\midrule	\textbf{Adequate} & Shows developing thinking about issue or goals & Provides explanation for choice of issue; questionable curiosity & Identifies the valid topic, but may be too broad or closed to investigation & Shows relation of issue to readings from class \\
\midrule	\textbf{Poor} & Presents topic/goals as list, not the result of thinking & Does not include valid or logical reasons for choice of issue & Fails to state subject as question to be answered or problem to be solved & Fails to cite related authors or relate issue to readings \\
	\bottomrule
\end{tabulary}
\end{table}
% section rubric (end)

\begin{comment}
	\section{Formatting} % (fold)
	\label{sec:formatting}
	While this document isn't your typical formal research report, you should get in the habit of formatting your documents to meet the guidelines of an appropriate style, such as the \textsc{mla} or \textsc{apa}. An \textsc{mla}-formatted template is available from Webcourses. Regardless, be sure your document meets these requirements:
	\begin{itemize}
		\item double-spaced lines,
		\item one-inch margins on all sides and half-inch indents for paragraphs,
		\item a 12-point typeface with serifs (like Times New Roman, \emph{not} Calibri), and
		\item parenthetical citations and a Works Cited or References page, as appropriate.
	\end{itemize}
	% section formatting (end)
\end{comment}

\end{document}
