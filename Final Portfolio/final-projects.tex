\documentclass[11pt,oneside]{amsart}	%defines this as an article
\usepackage{chrisfriend-comp} %provides formatting declarations for page, headers, figures, textcolor, comments, and bibliographic styles
\usepackage{chrisfriend-OTF-support} %provides support for OTF system fonts; incompatible with latex, rtf2latex, & ht4latex
%\usepackage[utf8]{inputenc} %support for smallamp?

%\usepackage{draftwatermark}

%\usepackage{tabularx}
\usepackage{tabulary} % allows for the tables I make rubrics with
%\usepackage{supertabular}
\usepackage{xtab} % allows tables to span pages
\usepackage{booktabs} % allows fancy lines in tables
\usepackage{rotating} % allows landscape tables
\usepackage{lscape} % allows rotated longtables
\usepackage{multirow} % allows rowspanning
\usepackage{enumitem} % helps with the overview
%\usepackage{paralist}

\title[Final Projects]{Assignment Sheet: Final Projects}
\chead{\scriptsize{\MakeUppercase{Final Projects}}}

\begin{document}
%\bibliographystyle{abbrv}
\thispagestyle{empty}

\vspace{-2in}
%\twocolumn[
\begin{center}
\huge
{\includegraphics[scale=.4]{pegasus.pdf}}

\textbf{Assignment Sheet: Final Projects \& Portfolio}

{\normalsize Chris Friend • \textsc{enc1102} • Summer B 2013}
\end{center}
\vspace{1.5\baselineskip}
%] %Use for column-spanning the title

\section{Purpose \& Goals} % (fold)
\label{sec:purpose}
We need to wrap up the semester. Because the emphasis of this course has been research, we will focus on research to draw things to a close. This project will demonstrate your accomplishments for the semester in several formats. Your first task is to revise your research proposal. The semester began with a research proposal that explained what you wanted to learn more about. Now that you have been studying that topic for several weeks, you know far more than you did when you initially created the proposal. It's time to put that new knowledge to good use and create an informed research proposal that more confidently argues your position and suggests the kind of research that would be appropriate and successful to follow the line of inquiry you began in this course. Revise or re-create your original research proposal to include a better sense of the research that already exist on the subject and insights you gathered from your own primary research.

To accompany your research proposal and present your ideas in a different and more easily digestible format, you need to create a pitch for your research plan. For this pitch, you will be using the Pecha Kucha format, which restricts presentations to twenty images displayed for twenty seconds each. Your job is to find a way to condense your ideas for the necessity and importance of your research down to only six minutes and forty seconds. Your mini presentation will be posted online, and the talks from our class will be gathered on our Tumblr page so you can see what everyone else has created.

That leaves us with the final portfolio and course audit. You've been focusing on the course outcomes after each assignment; now your goal is to package the work you've done in a way that showcases your accomplishments and highlights those course outcomes. Your traditional portfolio will include evidence of the course outcomes to show you've learned what you were supposed to learn in this course. Because we operate exclusively online, your portfolio will be, too. You can decide where your material is hosted\footnote{If you'd prefer not to post your portfolio online, you can instead create a digital portfolio that gets uploaded to Webcourses. See your instructor for details.}, but be sure the content is open-access and linked together. After assembling your portfolio, you'll create a Course Audit that explicitly states how and when you achieved the course outcomes, highlighting the parts of the portfolio that show evidence of your achievement. The Course Audit has its own assignment sheet, so be sure you review it before you start writing.
% section background (end)

\begin{table}[b]%\small
	\caption{Goals of Final Portfolio Components}\label{tab:goals}
\begin{tabulary}{\textwidth}{lLLL}
	\toprule  \textbf{\textsc{Component}} & \mbox{\textbf{\textsc{Audience}}} & \textbf{\textsc{Goals}} & \textbf{\textsc{App or Site to Use}}\\
\midrule	\textbf{Proposal} & Future teacher or funder & Convince people who support research that your project is well thought-out and ready to be approved/funded. & Word processor or \href{http://docs.google.com}{Google Docs} \\
\midrule	\textbf{Pitch} & Topic's stakeholders & Convince interested parties that more research needs to be done, and that your question is worth answering. & Presentation software, \href{http://www.pechakucha.org/}{Pecha Kucha}, \& \href{http://www.youtube.com/}{YouTube} or \href{https://vimeo.com/home/myvideos}{Vimeo} \\
\midrule	\textbf{Course Audit} & Friend \& DWR & Explain that you know what was expected of you and how you've met the expectations. & Word processor or \href{http://docs.google.com}{Google Docs} \\
\midrule	\textbf{Portfolio} & Friend \& DWR  & Provide and highlight evidence that you've achieved the course outcomes. & Word processor, \href{http://docs.google.com}{Google Docs}, \href{https://github.com/}{GitHub}, a blog, etc. \\
%\midrule	\textbf{Component} & Audience & Goals & Apps \\
	\bottomrule
\end{tabulary}
\end{table}


\section{Evaluation} % (fold)
\label{sec:evaluation}
Your performance on these final projects determines your score on the ``products'' portion of your final grade. Specific details for grading criteria appear in Table~\ref{tab:rubric}, but be sure to review all feedback you received from your instructor and your peers throughout the semester to ensure your final draft is as effective as possible.
% section rubric (end)

\section{Procedure} % (fold)
	\label{sec:procedure}
This assignment sheet covers several final activities, so be sure you understand each component before you start working. You may decide that the order presented below won't work for you. That said, I recommend completing tasks in this order:
\begin{description}
	\item[Revise Proposal] Revise the portfolio you wrote at the beginning of the semester. (This may feel more like rewriting it, considering you now know more than you thought you would back then.) Consider scheduling a conference to discuss your revision plans or progress with your instructor.
	\item[Build Pitch] Now, shift gears. After explaining why your research needs to be done using academic terms and an academic genre, now you need to do the same thing that for a different audience and with a different genre. Given 6\textonehalf{} minutes and exactly twenty slides, how would you justify the need for your proposed research project using only images and your voice, rather than text? Create a twenty-slide, Pecha Kucha-format presentation using presentation software of your choice. Then, record a video of the slides advancing every twenty seconds with your narration on top. Upload your video to a service like YouTube or Vimeo, then post a link on the course Tumblr blog so your classmates can see what you've created.
	\item[Assemble Portfolio]  This is actually the easy part. Now that your proposal is finalized, all you need to do is compile all of the documents you've created this semester into one manageable package. How you present that package is up to you: but I recommend sticking with what is simple and familiar to you. If your work is in Google Docs, just make a new document that includes links to existing ones, serving as a table of contents. If your work is on your computer, created with a word processor, use that word processor to create a table of contents\footnote{Ask your instructor for resources that show how to build a portfolio in a word processor.} that helps organize and present the work you have done. Feel free to check with your instructor if you have another idea for how to present your portfolio; this assignment is very flexible.
	\item[Write Course Audit] Now that you have collected all of your documents from the semester, you need to write a cover letter that articulates how those documents demonstrate that you have achieved the intended outcomes of the course. Write a letter to your instructor that helps guide him through a review of your portfolio, pointing out where you achieved each outcome listed on syllabus. Remember: the goal here is to demonstrate that you achieve the outcomes. Do not write a letter that emphasizes flattery or informs the instructor that this course has been your most breathtakingly life-changing experience to date, that you don't know what you would do without the amazing assistance or instructor gave the semester, etc. Although any of these situations may be true, pointing them out sounds more like kissing up and less like completing the assignment. Resist the urge.
	
	Add your course audit to your portfolio, preferably before the table of contents. This becomes your cover sheet.
	\item[Publish \& Submit Everything] Inside Webcourses, submit your work so that your instructor knows it is ready to be graded. For any work that is published elsewhere on the web (such as on YouTube or Google Docs), all you need do is submit a link that your instructor can follow. For documents that do not already live online (such as a portfolio created in a word processor), you will need to upload the document into Webcourses. For any work published online, please ensure access is open to anyone with a link.
	\item[Relax] Enjoy the feeling of being finished with the course!
\end{description}
% section procedure (end)

\begin{table}[b]%\small
	\caption{Evaluation of Final Products}\label{tab:rubric}
\begin{tabulary}{\textwidth}{rL}
	\toprule  & \textbf{\textsc{Demonstration of Course Outcomes}} \\
\midrule	\textbf{A} & Clearly demonstrates exceptional achievement of course outcomes and provides easily identifiable evidence of those outcomes across all submitted materials. \\
\midrule	\textbf{B} & Demonstrates strong mastery of course outcomes, made obvious through the content of all submitted materials. \\
\midrule	\textbf{C} & Shows attainment of course outcomes and provides documentation of those accomplishments through satisfactory attached documents. \\
\midrule	\textbf{D} & Provides evidence of an unsuccessful attempt to achieve the course outcomes. Supporting documents likely haphazard, of poor quality, or incomplete. \\
\midrule	\textbf{F} & Fails to demonstrate achievement of course outcomes or to submit a complete portfolio. Work is of unacceptably bad quality. \\
	\bottomrule
\end{tabulary}
\end{table}
% section rubric (end)

\end{document}