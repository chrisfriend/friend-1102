\documentclass[10pt,twocolumn]{amsart}	%defines this as an article
\usepackage{chrisfriend-comp} %provides formatting declarations for page, headers, figures, textcolor, comments, and bibliographic styles
\usepackage{chrisfriend-OTF-support} %provides support for OTF system fonts; incompatible with latex, rtf2latex, & ht4latex
%\usepackage[utf8]{inputenc} %support for smallamp?

%\usepackage{draftwatermark}

%\usepackage{tabularx}
\usepackage{tabulary} % allows for the tables I make rubrics with
%\usepackage{supertabular}
\usepackage{xtab} % allows tables to span pages
\usepackage{booktabs} % allows fancy lines in tables
\usepackage{rotating} % allows landscape tables
\usepackage{lscape} % allows rotated longtables
\usepackage{multirow} % allows rowspanning
\usepackage{enumitem} % helps with the overview
%\usepackage{paralist}
\usepackage{multicol}

\title[Course Audit]{Assignment Sheet: Course Audit}
\chead{\scriptsize{\MakeUppercase{Course Audit}}}

\begin{document}
%\bibliographystyle{abbrv}
\thispagestyle{empty}
\setlength{\columnsep}{.25in}

\twocolumn[
%\vspace{-2in}
\begin{center}
\huge
{\includegraphics[scale=.4]{pegasus.pdf}}

\textbf{Assignment Sheet: Course Audit}

{\normalsize Chris Friend • \textsc{enc1102} • Spring 2013}
\end{center}
\vspace{1.5\baselineskip}
] %Use for column-spanning the title

%\begin{multicols}{2}
\section{Background and Purpose} % (fold)
\label{sec:background}
Your last task for this course is to look back and review what you've accomplished. Consider what you have learned both explicitly and through your work. Think about the research process 
%(from brainstorming through proposal, bibliography, and analysis of potential outcomes) 
and the role that genre plays in writing.
%(from the sources at the beginning through academic papers in the middle and community-specific genres toward the end).
 What is your current understanding of the research process? What do you believe the role of genre is in the writing process?

Step back from your assignments and look at the ``big picture'' of the course. %Reflect on what you have learned, how your thinking has changed, and what your views of research and/or writing now are. 
Look back on the syllabus and note which Course Outcomes you have achieved. Then, write a letter to your instructor asserting that you achieved the course outcomes. This format affords more casual language than a formal essay, but it constrains you to a smaller number of pages. Consider what you think is worth emphasizing about your experiences this term.%\footnote{I plan to spend a few minutes in class brainstorming other affordances and constraints for the letter format. I'd also like to get my students to come up with either an organizational structure or an include/exclude list for content.}
% section background (end)

\begin{table}[b]
	\caption{Evaluation Rubric for Course Audit}\label{tab:rubric}
	\small
\begin{tabulary}{\columnwidth}{cLLL}
	\toprule  & \textbf{\textsc{Outcomes}} & \textbf{\textsc{Research}}\\
\midrule	\textbf{Excellent} 
& Connect outcomes to growth as writer, uses coursework for support 
%& Relates work on assignments to development of thinking as a writer or student 
& Evaluates how writer's work on research process connects with writing abilities or academic progress \\
\midrule	\textbf{Adequate} 
& Demonstrates achievement of course outcomes 
%& Discusses path of assignments during semester 
& Illustrates an understanding of research process \\
\midrule	\textbf{Poor} 
& Fails to address course outcomes 
%& Omits overview of course assignments 
& Omits discussion of research process \\
	\bottomrule
\end{tabulary}
\end{table}

\newpage
\section{Procedure} % (fold)
\label{sec:procedure}
\begin{description}
	\item[Brainstorm] Review the course outcomes listed on the syllabus. Which of the outcomes have you accomplished this semester? Which of them show your best work? What did you do in class that proves you met those goals?
	\item[Organize] Find specific examples of each course outcome in the material you have created for this course. Determine how your work in the course fits together to show your achievements.
	\item[Write] Draft a letter that explains how you met the outcomes and refers to the examples you came up with. This letter will be read by your instructor and possibly by a program assessment committee. In other words, your audience is familiar with the expectations and terminology used in Comp II at \textsc{ucf}.
	\item[Revise] Double-check your organization and style, and make sure you address each outcome and support yourself with specific, convincing evidence.
	\item[Compile] Assemble all the supporting material to prove your case. At a minimum, you should include the five major assignments you have already completed. If any minor assignments or other material better illustrates you achieved the course outcomes, include them, as well. Combine the material into a single document. A \textsc{pdf} is best.
	\item[Submit] Submit a compiled portfolio, including your Course Audit, to Webcourses.
%	\item[Relax] Smile, knowing there's nothing else you have to do for this class.
\end{description}
% section procedure (end)

\section{Evaluation} % (fold)
\label{sec:rubric}
%This assignment is not designed to evaluate your ability to kiss up to your instructor and claim that the class has immeasurably changed your life for the better. 
The focus of the assignment is to get you to take a broad view of the content and goals of the course. If you are able to address the course content and comment on your experiences within it, you will have done your job. The rubric included in Table~\ref{tab:rubric} provides evaluation details.
% section rubric (end)



% section rubric (end)
%\end{multicols}
\end{document}
