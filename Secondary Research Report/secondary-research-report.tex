\documentclass[11pt, twosides]{amsart}	%defines this as an article
\usepackage{chrisfriend-comp} %provides formatting declarations for page, headers, figures, textcolor, comments, and bibliographic styles
\usepackage{chrisfriend-OTF-support} %provides support for OTF system fonts; incompatible with latex, rtf2latex, & ht4latex
%\usepackage[utf8]{inputenc} %support for smallamp?

%\usepackage{draftwatermark}

%\usepackage{tabularx}
\usepackage{tabulary} % allows for the tables I make rubrics with
%\usepackage{supertabular}
\usepackage{xtab} % allows tables to span pages
\usepackage{booktabs} % allows fancy lines in tables
\usepackage{rotating} % allows landscape tables
\usepackage{lscape} % allows rotated longtables
\usepackage{multirow} % allows rowspanning
\usepackage{enumitem} % helps with the overview
%\usepackage{paralist}
\usepackage{color,soul}

\title[Secondary Research Report]{Assignment Sheet: Secondary Research Report}
\chead{\scriptsize{\MakeUppercase{Secondary Research Report}}}

\begin{document}
%\bibliographystyle{abbrv}
\thispagestyle{empty}

\vspace{-2in}
\begin{center}
\huge
\includegraphics[scale=.4]{pegasus.pdf}

\textbf{Assignment Sheet: Secondary Research Report}

{\normalsize Chris Friend • \textsc{enc1102} • Spring 2013}
\end{center}
\vspace{1.5\baselineskip}

\section{Background} % (fold)
\label{sec:background}
In your Research Proposal, you included a preliminary list of potential sources. These sources were to reflect the “conversation” surrounding the question or problem you identify to research.  By now, you should have reviewed the potential sources and found others to add to your research. This Secondary Research Report helps bring order to the chaos by organizing your sources, your findings, and your thinking on the problem or question you are researching.

% section background (end)

\section{Annotated Bibliography} % (fold)
\label{sec:annotated_bibliography}

\subsection{Purpose} % (fold)
\label{sec:bib-purpose}
You should have realized at some point that keeping track of the sources you are finding–and particularly what those sources are saying—quickly becomes overwhelming as you add more sources to your list. This is where an annotated bibliography comes in: it helps you organize your sources and keep track of your learning and your responses to everything you read. Conveniently, it also shows a curious instructor what you have learned from the conversation you have observed.
% section bib-purpose (end)

\subsection{Procedure} % (fold)
\label{sec:bib-procedure}
You have created bibliographies and Works Cited pages countless times. This assignment adds on to the traditional source list by including 200–300-word annotations (think \emph{summaries}) after each entry on the list. The annotations summarize the source, evaluate how the source relates to your research topic, and notes any significant contributions to your ongoing thinking. A sample Annotated Bibliography is available on Webcourses to give you a sense of the tone.

In short, your job is to use as few words as possible to provide as much information as possible about each source you have used, explaining its relevance to your research. To that end, perform these steps for writing your entries (see Figure~\ref{fig:sample-annotation-content} for an example):
\begin{enumerate}
	\item Create your correctly formatted list of \textbf{at least fifteen sources}. This is far easier said than done, so be sure to allow plenty of time to complete it. Repeat each step below for every item on your list.
	\item  Explain\label{step:source-id} the type and topic of each source, plus the qualifications of the author.
	\item  Describe the research methods used (if any). Summarize the main claims of the source. Include as much detail as needed to make the annotation a useful resource for you to refer back to. Direct quotations of particularly useful information are always appropriate; be sure to note page numbers.
	\item  Resist the urge to bs. It will not help your research, and your instructor does not want to read it.
	\item  Clarify how the source is connected to your research.
	\item  Critically evaluate the claims and credibility of each author. If you believe the author is biased, or if you think the data (or their interpretation) are flawed, explain why.
\end{enumerate}


\begin{figure}[ht]	
	\centering
	\fbox{
	\begin{minipage}{.75\textwidth}
		In this scholarly journal article, John Jones, a researcher at Johns Hopkins University, evaluates approaches to reading incentives. Jones surveyed 400 high school students and concluded that A and B types of incentives are not as useful as C types of incentives. This article is of relevance to my research project in its definitions of motivation and incentives and in its findings about specific incentive programs. Jones argues that Pizza Hut’s BookIt program is the only successful reading incentive program. However, his claim is suspect because he is a shareholder in Pizza Hut stock.
	\end{minipage}
	} % fbox
	\caption{Sample structure (not formatting or length) of an annotation.}
\label{fig:sample-annotation-content}
\end{figure} % annotation content
% section bib-procedure (end)

\subsection{Evaluation} % (fold)
\label{sec:bib-eval}
Because this list of sources shows the progress you have made so far in your research and shows that you can select relevant and focused sources, you will primarily be evaluated on the quality of your selected sources. Sources that are not on-topic or that seem to be whatever you found the night before deadline will severely limit the thinking you can do and connections you can make as you explore your question or problem. You must also show your ability to read critically and succinctly summarize each source. Finally, your citation formatting needs to consistently follow the expectations of whichever style you are using. Grading guidelines are outlined in Table~\ref{tab:bib-rubric}.
% section bib-eval (end)

\begin{table}[b]
	\caption{Evaluation of Annotated Bibliography}\label{tab:bib-rubric}
\begin{tabulary}{\textwidth}{rLLL}
	\toprule  & \textbf{\textsc{Quality and Variety}} & \textbf{\textsc{Summary}} & \textbf{\textsc{Citation Style}} \\
\midrule	\textbf{Excellent} & Sources provide diverse range of perspectives and thoroughly cover research topic & Annotations explain the context of each source and show direction of research & Citations are expertly formatted and meticulous \\
\midrule	\textbf{Adequate} & Sources are varied and relevant; range and coverage may be lacking & Point and purpose of each source are clear; context may be lacking & Citations are generally accurate and decently formatted \\
\midrule	\textbf{Poor} & Sources are limited in scope, narrow, or incomplete & Annotations failed to explain relevance or content of sources & Citations are lacking, sloppy, or inaccurate \\
	\bottomrule
\end{tabulary}
\end{table}
% section bib-eval (end)

% section annotated_bibliography (end)

\section{Framing Synthesis} % (fold)
\label{sec:framing_synthesis}

\subsection{Purpose} % (fold)
\label{sec:synth-purpose}
The \nameref{sec:framing_synthesis} takes the individual documents from your bibliography, combines them into a cohesive whole, and demonstrates how they work together to shape or direct the question or problem you are pursuing. In this part of your Secondary Research Report, you should emphasize showing relationships among the sources you found and connecting them to the major issue at stake. Ultimately, this synthesis will show what you have learned through your research and what you suspect the final resolution may look like.
% section synth-purpose (end)

\subsection{Procedure} % (fold)
\label{sec:synth-procedure}
In this 2–3-page analysis, you are presenting your understanding of the current state of the “conversation” you found while creating your annotated bibliography. \textbf{Synthesize} the various sources, rather than summarize them; you summarized in your annotations. If certain sources stand out in your mind as being more important than others, be sure that comes across in your synthesis. As you write, consider the following questions as suggestions to expand your thinking:
\begin{itemize}
	\item What connections did you find among the sources you chose?
	\item Where do the disagreements come from, and who are the loudest speakers?
	\item What are the main findings from among the contributors?
	\item  What have you learned that has changed or expanded the question/problem you are researching?
	\item  How close do these sources come to answering the question \textbf{or} addressing the problem?
\end{itemize}

Note: While your \nameref{sec:framing_synthesis} can only be written \emph{after} you create your \nameref{sec:annotated_bibliography}, it should appear \emph{before} the list of sources in your submitted document.
% section synth-procedure (end)

\subsection{Evaluation} % (fold)
\label{sec:synth-rubric}
This component of your analysis exists to show how well you can connect the sources you have found and relate them to the topic you are researching. Evaluation will emphasize depth and thoroughness of thinking, specifically in terms of your ability to synthesize the sources into a coherent conversation. Specific details regarding assessment are in Table~\ref{tab:synth-rubric}.

\begin{table}[b]
	\caption{Evaluation of Framing Synthesis}\label{tab:synth-rubric}
\begin{tabulary}{\textwidth}{rLLLL}
	\toprule  & \textbf{\textsc{Depth \& Progress}} & %\textbf{\textsc{Selection and Variety}} &
	 \textbf{\textsc{Synthesis}} & \textbf{\textsc{Relevance \& Connections}}\\
\midrule	\textbf{Excellent} & Competent \& deliberate thinking on problem/question & Researcher naturally \& fluently integrates disagreeing sources into one explanation & Sources shown to speak cohesively and directly to research topic \\
\midrule	\textbf{Adequate} & Problem/question clearly defined; choppy discussion & Clear explanation of viewpoints; lacks smooth integration & Sources shown to relate to topic, perhaps indirectly or disjointedly \\
\midrule	\textbf{Poor} & Author uncomfortable or unfamiliar with research topic & Source perspectives neither connected nor set in opposition & Sources not tied together as having one conversation; soloists, not a chorus \\
	\bottomrule
\end{tabulary}
\end{table}
% section synth-eval (end)

% section framing_synthesis (end)
\end{document}

\section{Formatting} % (fold)
\label{sec:formatting}
This document should follow standard formatting requirements for the citation style you are using, such as \textsc{mla} or \textsc{apa}. An \textsc{mla}-formatted template is available from Webcourses. Be sure your document includes:
\begin{itemize}
	\item double-spaced lines, including the entire bibliography, with no extra spacing between entries;
	\item one-inch margins on all sides and half-inch indents for paragraphs;
	\item a 12-point typeface with serifs (like Times New Roman, \emph{not} Calibri); and
	\item parenthetical citations in your synthesis, where appropriate.
\end{itemize}
% section formatting (end)

