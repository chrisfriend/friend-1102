\PassOptionsToPackage{table}{xcolor}
\documentclass[11pt,oneside,draft]{amsart}	%defines this as an article
\usepackage{chrisfriend-comp} %provides formatting declarations for page, headers, figures, textcolor, comments, and bibliographic styles
\usepackage{chrisfriend-OTF-support} %provides support for OTF system fonts; incompatible with latex, rtf2latex, & ht4latex
%\usepackage[utf8]{inputenc} %support for smallamp?

%\usepackage{tabularx}
\usepackage{tabulary} % allows for the tables I make rubrics with
%\usepackage{supertabular}
\usepackage{xtab} % allows tables to span pages
\usepackage{booktabs} % allows fancy lines in tables
%\usepackage{rotating} % allows landscape tables
\usepackage{lscape} % allows rotated longtables
\usepackage{multirow} % allows rowspanning
%\usepackage[table]{xcolor}    % loads also »colortbl«; allows for shaded rows
\usepackage{enumitem} % helps with the overview
%\usepackage{paralist}
%\usepackage{draftwatermark}

\usepackage{acronym}
\acrodef{ucf}[\textsc{ucf}]{the University of Central Florida}
\acrodef{dwr}[\textsc{dwr}]{the Department of Writing and Rhetoric}
\acrodef{waw}[\textsc{waw}]{Writing About Writing}
\acrodef{uwc}[\textsc{uwc}]{University Writing Center}
\acrodef{1102}[\textsc{enc~1102}]{Composition II}
\acrodef{rr}[\textsc{rr}]{Reading Response}


\title[\textsc{enc}~1102 Syllabus]{Course Syllabus: Composition II}
\chead{\scriptsize{\fontspec[Numbers=Lining]{Garamond Premier Pro}\MakeUppercase{ENC~1102 Syllabus}}}

  
\begin{document}
%\bibliographystyle{abbrv}
\thispagestyle{empty}
\vspace{-2in}
\begin{center}
\huge
\includegraphics[scale=.4]{pegasus.pdf}

\textbf{Course Syllabus: Composition II}
\end{center}


\label{sec:about_the_course}
\vspace{1.5\baselineskip}
\begin{center}
\begin{minipage}{0.66\textwidth}
%	|\hfill|
	\begin{description}[align=right, labelwidth=*, labelindent=0.9in, leftmargin=1in]
%	\item[Course Title] Composition II
	\item[Prerequisite] Successful completion of \textsc{enc} 1101 or have passed an AP English Exam
	\item[Meeting]  Online (\textsc{enc} 1102.BW65)
	\item[Term] Summer B 2013
	\vspace{0.5\baselineskip} \hrule \vspace{0.5\baselineskip}
	\item[Instructor] Christopher R. Friend
	\item[Email] \href{mailto:friend@ucf.edu}{friend@ucf.edu}
	\item[Office] Colbourn Hall (\textsc{cnh}) 306 \textsc{D}\todo{Verify new room number}
	\item[Office Hours] In person or online, but by appointment. \\ Visit \href{http://friend.lattiss.com}{http://friend.lattiss.com} for available times or \href{mailto:friend@ucf.edu}{email instructor} for other options.
\end{description}
\end{minipage}
\end{center}
\vspace{0.75\baselineskip}

\section{Overview} % (fold)
\subsection{Course Description} % (fold)
\label{sub:course_description}
In this course, we will explore the research process as one of genuine inquiry. During the semester, you will be expected to:
\begin{itemize}
	\item ask challenging, open-ended questions requiring inquiry to answer;
	\item read carefully what others have said about those questions; 
	\item create new knowledge about those questions using appropriate primary and secondary research methods, including library research, historical analysis, rhetorical analysis, survey, or interview; and
	\item join the existing research ``conversation'' on relevant topics.
\end{itemize}
You will use writing as a tool to help you learn each of the concepts listed above, and you will become more aware of your style and abilities as a researcher.
% subsubsection my_version (end)
% subsection course_description (end)


% section about_the_course (end)
\subsection{Course Outcomes}\label{outcomes}
Through successful completion of this course and its activities, you should be able to
\begin{itemize}
	\item read, analyze, and respond to difficult texts;
	\item understand texts as claims and test those claims;
	\item ask meaningful questions about the literacies required in the 21st century;
	\item use research and analysis to seek answers to those questions;
%	\item gather and analyze data of various kinds;
%	\item use technologies to help achieve writing and research goals;
%	\item thoughtfully discuss 
	\item convey written ideas and research findings effectively for various audiences and purposes; and
	\item explain writing-related concepts, including \emph{intertextuality}, \emph{genre}, \emph{originality}, \emph{plagiarism}, and \emph{the technologies of writing and research}.
\end{itemize}


\section{Required Materials}
\begin{itemize}
%\item \citeauthor{greene:2008aa}, \citetitle{greene:2008aa}\\ (\textsc{isbn} \href{http://www.worldcat.org/search?q=978-0-312-60140-9}{978-0-312-60140-9}) %(Bring to class each day.)
%\item \citeauthor{lunsford:2010aa}, \citetitle{lunsford:2010aa}, \emph{Fourth Edition}\\ (\textsc{isbn} \href{http://www.worldcat.org/search?q=978-0-312-66486-2}{978-0-312-66486-2})%(Bring only when requested.)
\item A reliable connection to the Internet. Please have backup plans/devices available. See \refname{
\end{itemize}

\section{Grading \& Assessment}
Your grade in this course will be based on two holistic grades listed in Figure~\ref{tab:assignments}. Think of these like grades for a semester-long portfolio: the components work together to build the overall value of the whole. That value is what will be graded in this course. You will get consistent feedback throughout the semester to help ensure you are on-track for a successful grade. Additionally, each major project will have a specific assessment rubric, and every smaller assignment will have detailed completion guidelines that will be provided in class and available on Webcourses.  The smaller assignments are designed to help you build toward your final project and should not be dismissed.

Please note the following distinctive characteristics about grading in this course:
\begin{itemize}
	\item You can earn a D for an assignment or major component, but you cannot earn a D for this course. To pass, you must earn at least a C average.
	\item The grade of NC (no credit) can be assigned at the instructor’s discretion only if you complete all course work on time, attended class regularly, and fail to produce satisfactory work for the class.
\end{itemize}


\subsection{Grading Standards} % (fold)
\label{sub:grading_standards}
Regular attendance, participation in all activities, and successful completion of all assignments (as defined by each assignment's assessment rubric) will earn you a passing grade of C, indicating that you have achieved the expected outcomes of the course. If you do not take part in all assignments and activities, you should not expect a passing grade in the course. If the quality of your work or your participation falls below acceptable standards (i.e. if you are heading for failure), I will be sure to let you know. Along more optimistic lines, grades of B or A are used for work that is good and excellent, respectively, surpassing the basic expectations. Assignment sheets will suggest ways to exceed those expectations, so you won't have to guess. If your performance exceeds basic standards, I will be sure to let you know.

Besides participation, all grades for this course come from products turned in at the end of the semester (see Figure~\ref{tab:assignments}). This is by design, to allow you the chance to experiment and take risks as you progress through the course. If your research leads you down a dead-end, your grade will not be affected. In lieu of grades, you will receive constant feedback from your peers and instructor about what is and isn't working well with your research and various written products. Take the feedback seriously, and revise your work regularly throughout the semester so that your final portfolio best reflects your ability.
% subsection grading_standards (end)

\begin{figure}[t] 
	\centering
	\subtable[Grade Calculations]{\small
		\noindent\begin{tabulary}{0.25\textwidth}{lr}
		\toprule\textbf{\textsc{Grade}} & \textbf{\textsc{Min. Points}}\\
		\midrule
			A	&	185	\\
			A-	&	180	\\
			B+	&	175	\\
			B	&	165	\\
			B-	&	160	\\
			C+	&	155	\\
			C	&	145	\\
			C-	&	140	\\
			NC	&	Unsatisfactory	\\
			F	&	Partial/Poor	\\
		\bottomrule
		\end{tabulary}\label{tab:final-grades}
	} %subtable
	\quad % Middle gap spacing
	\subtable[Grade Distribution]{\small
		\noindent\begin{tabulary}{\textwidth}{Lr}
			\toprule\textbf{\textsc{Assignment}} & \textbf{\textsc{Points}}\\
				\midrule	Products (portfolio \& genre presentation)	&	100	\\
%				\midrule	Academic Research Report	&	25	\\
%				\midrule	Genre Product \& Presentation	&	25	\\
				\midrule	Process (participation \& collaboration)	&	100	\\
				\midrule	\textbf{\textsc{Total}}	& \textbf{\textsc{200}}\\
			\bottomrule
		\end{tabulary}\label{tab:assignments}
	} %subtable
	\caption{Course Grading System}
\end{figure}


\subsection{Expectations} % (fold)
\label{sub:expectations}
This class will employ collaboration to a larger extent than is typical in a composition course. You will not be doing group projects, but you will be working with one another to learn from one another's work, progress, and feedback.

While enrolled in this course, you can expect these things from me:\footnote{The structure and approach of the Expectations section is adapted from the \href{http://www.ceball.com/classes/239/spring09/?page_id=8}{English 239 syllabus} of Cheryl E. Ball, \textsc{isu}.}
\begin{itemize}
	\item enthusiasm for research, teaching, and writing;
	\item clarity and thoroughness in assignments, goals, and expectations;
	\item personal interest in your learning and work;
	\item freedom to be creative with the products you create for this course;
	\item critical feedback to help you improve your thinking and writing; and
	\item preparation to ensure a beneficial and productive semester.
\end{itemize}
If at any point you feel I am failing to meet any of those expectations, please let me know. Your feedback is the best way I can learn how to improve my teaching.

As we progress through the semester, your peers and I will expect these things from you:
\begin{itemize}
	\item consistent attendance and active participation in class activities, both online and on-ground;
	\item informed contributions, based on sufficient preparation and consideration (i.e. doing the readings and research)
	\item an open mind, tolerant and curious about differences of opinion; and
	\item honest and polite commentary and feedback that helps your peers improve their work.
\end{itemize}

During class discussions and as you work on your assignments, keep in mind that I value these things in my students:
\begin{itemize}
	\item thought-out and supported opinions;
	\item willingness to take risks and try new approaches to solving problems, as risks often create the greatest opportunities;
	\item creativity in how you respond to the challenges created and faced by this course; and
	\item excellence in your work, showing the best you can produce.
\end{itemize}

% subsection expectations (end)


\section{Major Assignments} % (fold)
This course builds to a final project designed to persuade people to take action on a literacy-related issue (broadly defined) of your choosing. Each unit of study (see Figure~\ref{fig:vis-syllabus}, \nameref{fig:vis-syllabus}) includes a major writing assignment designed to form, refine, and express your ideas toward that final project. For each assignment, you will receive a detailed assignment sheet with specific procedures, expectations, and an evaluation rubric. An abridged overview of each major assignment appears below.

\begin{figure}[t]
	\centering
		\includegraphics[width=\textwidth]{visual-syllabus.pdf}
	\caption{Course Assignments Overview}
	\label{fig:vis-syllabus}
\end{figure}


\subsection{Brainstorming Audit} % (fold)
\label{sub:brainstorming_reflection}
	\begin{description}
	\item[Task] Choose a potential issue or question you would like to research, based on the readings at the beginning of the semester. Document your decision and what led you to it.
	\item[Purpose] Show that you can:
		\begin{enumerate}
		\item identify a relevant, researchable problem;
%		\item determine and explain why you are interested in that problem;
		\item specify what you wish to learn about the problem; and
		\item document and properly cite the readings that led to your decision.
		\end{enumerate}
	\end{description}
% subsection brainstorming_reflection (end)

\subsection{Research Proposal} % (fold)
\label{sub:research_proposal}
	\begin{description}
	\item[Task] Create a plan for conducting a semester-long research study.
	\item[Purpose] Show that you can:
		\begin{enumerate}
		\item identify a clear research problem or question,
		\item create a plan of action for exploring your chosen issue,
		\item show that your study is important and relevant, and
		\item suggest a target audience for your findings.
		\end{enumerate}
	\end{description}
% subsection research_proposal (end)

\subsection{Secondary Research Report} % (fold)
\label{sub:context_analysis}
This assignment has two components that combine to document the situation and existing knowledge in which your research is taking place.
\subsubsection{Annotated Bibliography} % (fold)
\label{ssub:annotated_bibliography}
	\begin{description}
	\item[Task] Create a list of sources related to the issue you are investigating.
	\item[Purpose] Show that you can:
		\begin{enumerate}
		\item identify a variety of sources related to your issue,
		\item describe other people's research methods and claims, and
		\item evaluate the validity of claims and arguments.
		\end{enumerate}
	\end{description}
% subsubsection annotated_bibliography (end)
\subsubsection{Framing Synthesis} % (fold)
\label{ssub:framing_synthesis}
	\begin{description}
	\item[Task] Present your collection of sources as a cohesive whole.
	\item[Purpose] Show that you can:
		\begin{enumerate}
		\item explain the relationships among sources,
		\item relate your sources to the question or problem you are researching,
		\item identify the shape and nature of present conversation around the issue.
		\end{enumerate}
	\end{description}
% subsubsection framing_synthesis (end)
% subsection context_analysis (end)

\subsection{Primary Research Report} % (fold)
\label{sub:outcomes_analysis}
This assignment has three components that combine to move the existing knowledge on a topic forward, growing from the positions identified in the Secondary Research Report.
	\begin{description}
	\item[Task] Create knowledge about your chosen issue; determine how that information would best be used.
	\item[Purpose] Show that you can:
		\begin{enumerate}
			\item gather new knowledge about your chosen issue through appropriate primary research,
		\item identify the people involved in the conversation,
		\item examine their relation to the issue in a stakeholder analysis,
		\item determine what genres are used by those stakeholders, and
%		\item evaluate which genre would be most effective for your final project.
 		\item make decisions regarding your chosen issue based on the items above.
		\end{enumerate}
	\end{description}
% subsection outcomes_analysis (end)

\begin{comment} %Not using this assignment in a compressed semester
	\subsection{Academic Research Report} % (fold)
	\label{sub:academic_research_report}
		\begin{description}
		\item[Task] Write a formal paper arguing your position regarding the topic you chose to research.
		\item[Purpose] Show that you can:
			\begin{enumerate}
				\item synthesize the material you have found and created into a coherent argument;
				\item support your argument with extended reasoning, logically organized;
				\item use language rhetorically appropriate to an academic setting;
				\item manage citations and references appropriately to support your argument in context.
			\end{enumerate}
		\end{description}

	% subsection academic_research_report (end)
\end{comment}

\subsection{Final Portfolio} % (fold)
\label{sub:final_portfolio}
This assignment combines the major assignments listed above into a single document that reflects your progress through the semester. The portfolio includes a Course Audit that serves as a cover letter, reflecting on the semester and directing readers to the accomplishments seen in your work. This assignment will:
	\begin{enumerate}
	\item demonstrate how your research this semester has met the desired \nameref{outcomes} and
	\item show self-awareness of your writing and research practices.
	\end{enumerate}
% subsection final_portfolio (end)

\begin{comment}% Omitting in the condensed semester
	\subsection{Genre Product} % (fold)
	\label{sub:final_project}
		\begin{description}
		\item[Task] Use a relevant genre to communicate with your stakeholders to convince them to take an action or make a decision.
		\item[Purpose] Show that you can:
			\begin{enumerate}
			\item effectively use the selected genre,
			\item use proper appeals to communicate with your stakeholders, and
			\item present a polished final product to your peers using appropriate rhetoric.
			\end{enumerate}
		\end{description}
	% subsection final_project (end)
\end{comment}
% section major_assignments (end)




%\begin{comment} %%%%%%%%%%%%%%%%%%%%%%% Calendar


%\begin{landscape}
%\clearpage
%\addtocounter{page}{-2}
\section{Course Calendar}
\todo[inline]{Update calendar workfile; insert data here.}
%\addtocounter{page}{-1} %accommodates the blank page before Calendar
%\begin{center}
{\centering\footnotesize %brace balanced at end of calendar
%\vspace{-1in}
\tablehead{\toprule\textbf{\textsc{Unit}} & \textbf{\textsc{Week}} & \textbf{\textsc{Date}} & \textbf{\textsc{Readings/Homework\newline(Before Class)}} & \textbf{\textsc{Guiding Question\newline(During Class)}}\\}
\tablelasthead{\toprule\textbf{\textsc{Unit}} & \textbf{\textsc{Week}} & \textbf{\textsc{Date}} & \textbf{\textsc{Readings/Homework\newline(Before Class)}} & \textbf{\textsc{Guiding Question\newline(During Class)}}\\}
\begin{mpxtabular}{>{\bfseries}p{0.75in}ccp{1.8in}p{1.85in}} % for portrait
%\begin{xtabular}{>{\bfseries}lccp{2.25in}p{3.25in}} % for landscape
\midrule
%	\toprule\textbf{\textsc{Topic(s)}} & \textbf{\textsc{Week}} &\textbf{\textsc{Class Discussion}} & \textbf{\textsc{Readings/Homework}}\\

% Do not edit manually. Paste content in from Calendar Workfile. (Numbers document, this folder.)







 	\bottomrule
    \end{mpxtabular}
} %matches the \footnotesize at top of calendar
%    \end{center}


  
\subsection{Changes}
    Material in the preceding schedule is subject to change at the discretion of the instructor.  Students will be notified of any changes in class.  If relevant, changes will also be reflected on Webcourses.
%    \end{landscape}

\subsection{Final Exams} % (fold)
\label{sub:final_exams}
Because this class includes a portfolio that documents your progress over the semester, as well as a project you create as a result, there is no final exam as such. However, class will meet on exam day to celebrate our accomplishments and share our projects. Your project presentations during exam periods take place in the regular classroom at these times:
\begin{itemize}
	\item 9:30 class: Friday, 26 April 2013, 7:00 a.m. to 9:50 a.m.
	\item 10:30 class: Monday, 29 April 2013, 10:00 a.m. to 12:50 p.m.
\end{itemize}
% subsection final_exams (end)
  


%\end{comment} %%%%%%%%%%%%%%%%%%% Calendar









\section{Policies \& Miscellanea}
%\nocite{Curtis:2009uq,Tripp:2009kx,Wardle:2010fk} %ensures the syllabi I stole from will be in the Works Consulted list

\subsection{Attendance}
\todo{Revise for online}Your attendance is mandatory, and your success in this course depends on your active engagement.  If you are absent more than three times, I will recommend that you drop the class; more than six times, and you risk failing the course.  If you must be absent, it is \emph{your} responsibility to complete the day’s activities and contact your peers to determine what you missed and how you need to recover. Any absence will cause you to forfeit the points for any participation or activities for those days. (Note that because major papers are collected online, absence from class will not affect the deadline or score for online submissions.)

Absences due to University-sponsored events—such as music performances, athletic competitions, debates, and some conferences—can excuse you from certain minor assignments (but not major papers). When participating in school-sponsored events, submit a Program Verification form to your instructor no later than the day you return to class. Absences due to religious holidays should be discussed with the instructor during the first week of the semester.

Please note that major assignments will be submitted online, so attendance (or lack thereof) does not affect your ability to submit work. You are still expected to turn in your work regardless of whether you are in class that day.

\begin{comment}
	\subsection{Late Work}
	It is in your best interest to \emph{not} get behind in this class; getting caught up is quite difficult. Thus, I generally do not accept late work for credit. If you are experiencing special circumstances and would like to request an extension, speak with me \emph{well in advance} of the due date and use all the rhetorical strategies at your disposal to effectively make your case. I reserve the right to accept late work with or without penalty only as the circumstances warrant. 
\end{comment}

\begin{comment}
	\subsection{Grades}
	\todo{Update to reflect no-grading approach.}Your instructor uses a ten-point grade scale in this course. For all graded assignments, a ``C'' is average and indicates you have acceptably completed all requirements.  A grade below ``C'' indicates you have not met minimum standards.  ``B'' is an honor grade, awarded for work that is thoughtful and well-written. ``A'' work is excellent and goes beyond what is required.  A grade of ``D'' on an assignment indicates that the work partially fails to meet the standards of the assignment, whereas an ``F'' indicates overall failure to meet the expectations.
\end{comment}

\subsection{Gordon Rule} % (fold)
\label{sub:gordon_rule}
\ac{1102} is a Gordon Rule class, meaning that you will be writing at least four major assignments, and you must earn a ``C'' or better to earn credit for the course. The assignments that contribute to your final portfolio meet this Gordon Rule requirement.
% subsection gordon_rule (end)



\subsection{Etiquette}
In short, the members of this class, both the instructor and the students, are expected to behave courteously and professionally in all interactions.  Under that umbrella statement, the following general guidelines should be followed in any class here at \ac{ucf}.
	\begin{description}
	\item[Tolerance] Many of our discussions in class will be driven by opinions and based on challenging material.  Since we are all writers, everyone in class will have personal experiences and viewpoints that can contribute meaning to the conversations.  All participants are expected to treat others with dignity and respect and are expected to refrain from insensitive comments, including racist, ageist, sexist, classist, homophobic, or other disparaging and unwarranted views.
	\item[Timeliness] Students are expected to be ready for class at its designated time just as much as you expect the instructor to dismiss class by the designated time.  Should you arrive to class late for any reason, please do so with a minimum level of disruption.  If you need to leave class early for any reason, please notify the instructor in advance and be as non-disruptive as possible when leaving.
	\item[Cellphones] As a courtesy, all cellphones should be silenced during this or any other class. Should your phone accidentally create a distraction during class, you should take action to eliminate the distraction without adding to it.
	\item[Computers] You will need to use your computer in class regularly to collaborate with others and complete your assignments. Having the discipline of shutting off distractions (such as Facebook, chat applications, etc.) improves your ability to focus and participate meaningfully.
	\item[Messages] Grammar, spelling, and punctuation reflect your abilities as a writer.  Keep in mind that emails and discussion posts you write for this class are being read by an English teacher in a composition course.  I do expect you to use care and consideration when corresponding, even if the intended audience is your peers. As a courtesy to any recipient, any email you send should begin with a salutation and end with a closing.  If asking a question about an assignment, be sure to indicate which assignment you are asking about.  Your instructor tends to be scatter-brained and appreciates the clarity.
	\item[Email] As a \ac{ucf} student, you have access to a Knights Mail account, which will be the primary method of communication for course-related announcements and information. You should check your mail at least three times a week; daily is preferable. Your instructor strives to reply to messages within 24 hours. If you email him a question, be sure to check for a response before your next class meeting.
	\end{description}
	

\subsection{Computer Reliability}\label{sub:reliability}
Save everything, and save often.  Computer problems are regular part of life, and I expect you to prepare for them rather than use them as an excuse for late work. Every semester, your instructor has had at least three students sustain a complete hard drive failure, losing all their work. Such failures are unpreventable, but they are recoverable, if you plan ahead. Working backups and protection from Windows viruses are essential to avoid the most common catastrophes.  A free Dropbox account (\href{http://db.tt/mzWxY8s}{http://dropbox.com}) provides convenient and automatic backups, allows you to access your files from any networked computer in case disaster befalls yours, and preserves old versions of files so that if a file is deleted or altered, a previous copy can be restored.

\subsection{Helpful Resources} % (fold)
\label{sub:helpful_resources}
\subsubsection{Writing Center}\label{ssub:uwc}
The \ac{uwc} provides free help for students writing papers for class.  Consultations (which can be in-person or online) can help with planning, drafting, or revising your papers.  Consider using the \ac{uwc}'s services, particularly in the early stages of planning a document.  Learn more at \href{http://uwc.ucf.edu/}{http://uwc.ucf.edu} or by calling 407-823-2197.  Please note that the weeks of midterms and finals can be very busy there; you are strongly encouraged to make a reservation. A link to the \ac{uwc} appointment scheduler is available in Webcourses.

\subsubsection{Knowledge Commons (in the library)} % (fold)
\label{ssub:knowledge_commons_in_the_library_}
\todo{Write this content.}
% subsubsection knowledge_commons_in_the_library_ (end)

\todo{Add sub§s for SDS, Counseling?}
% subsection helpful_resources (end)

\subsection{Plagiarism}
Students at \ac{ucf} are expected to act with integrity, in terms of both classroom behavior and intellectual property.  For details, please see the \href{http://www.goldenrule.sdes.ucf.edu}{Golden Rule Student Handbook}, section \textsc{ucf}-5.008.1.e.  Violations of this ethical cornerstone will result in disciplinary action, which can include any of the following:
\begin{itemize}
	\item loss of credit on an assignment
	\item a ``Z grade'' for the course (see \href{http://z.ucf.edu}{http://z.ucf.edu} for details)
	\item loss of credit for the course
	\item removal from the University
\end{itemize}
In an effort to protect the integrity of your work and ensure it is not re-used by others later, your instructor may ask that your assignments be submitted to Turnitin.com by their deadlines.
%\todo{Am I sure? Do I want to do this? Can it be integrated with Webcourses?}  

In this course, we will be discussing the use of outside texts for writing in and out of the classroom, specifically the use of source documentation/citation/attribution. If you have questions about correct documentation of sources, consult a writing handbook (such as \citetitle{lunsford:2010aa} by Andrea Lunsford), the style guide for the citation system you are using (such as \citetitle{gibaldi:2009aa} by Joseph Gibaldi), the \ac{uwc} (see Section~\ref{ssub:uwc}), or your instructor during office hours. Use of outside sources without proper credit, turning in work that is not your own, or assisting others to do either are each considered plagiarism and are subject to the above consequences.

\subsection{Accommodations}
At \ac{ucf}, we are committed to providing reasonable accommodations for all persons with disabilities. Students with disabilities who need accommodations in this course must contact the instructor at the beginning of the semester to discuss needed accommodations. No accommodations will be provided until the student has 1) registered with Student Disability Services (\textsc{src} Room 132, phone 407-823-2371, or \textsc{tty/tdd} 407-823-2116), and 2) met with the instructor to request accommodations.

More personally, I am dedicated to incorporating inclusive practices for all students within the classroom, as well as providing for specific accommodation requests. Beyond the provisions of Student Disability Services, please feel free to contact me with any suggestions and/or requests you have regarding the accessibility of information and/or interactions in this course. I am always interested in these types of suggestions, as they may not only meet a specific student's needs, but could be employed to make the overall class more accessible and inclusive for all students.\footnote{The second ¶ in the ``Accommodations'' section is adapted from the syllabus of Barbi Smyser-Fauble, \textsc{isu}.}

\subsection{\textsc{ucf} Allies} % (fold)
\label{sub:allies}
Your instructor is a \ac{ucf} Ally for the lesbian, gay, bisexual, transgendered, and questioning (\textsc{lgbtq}) community on campus. All \ac{ucf} Allies offer acceptance, support, and a safe space for anyone who is \textsc{lgbtq} or is working with issues of sexual identity. Allies answer questions and hold discussions in an open and non-judgmental way, and they can refer you to campus and community resources, as needed. All \ac{ucf} Allies have attended a training workshop to learn about about oppression, heterosexism, homophobia, the coming out process, and the benefits and responsibilities of being an Ally. Your instructor occasionally helps facilitate these workshops, so you are especially welcome to reach out to him to discuss any related issues. Feel free to visit during office hours or contact him by email. For more information about the \ac{ucf} Allies program, visit \href{http://allies.sdes.ucf.edu/faq}{http://allies.sdes.ucf.edu/faq}.
% subsection allies (end)

\subsection{Instructor's Research} % (fold)
\label{sub:instructor_s_research}
For the purposes of conducting research or improving his teaching practices, your instructor may use your work anonymously as an example in other classes, in workshops and lectures, or in publications. For example, I might quote from one of your assignments in a journal article or conference presentation, without revealing your identity. If you do \textbf{not} wish your work to be used in this manner, let me know in writing (via email is fine) within one week after the date your final grade is available. (This date is listed on \href{http://registrar.sdes.ucf.edu/calendar/academic/}{\ac{ucf}’s Academic Calendar}.) Your course grade will not be affected by your decision to permit or deny my use of your work. You can ensure my impartiality by notifying me after the date grades are due, which is also listed on \href{http://registrar.sdes.ucf.edu/calendar/academic/}{\ac{ucf}’s Academic Calendar}.\footnote{The ``Instructor's Research'' section is adapted from the syllabus of Beth Rapp-Young, \textsc{ucf}.}
% subsection instructor_s_research (end)


\section{Works Cited}\label{works-con}
\renewcommand\refname~{\vspace{-22pt}}

\printbibliography

\end{document}
